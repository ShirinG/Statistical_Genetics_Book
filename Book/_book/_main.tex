\documentclass[12pt,]{book}
\usepackage{lmodern}
\usepackage{amssymb,amsmath}
\usepackage{ifxetex,ifluatex}
\usepackage{fixltx2e} % provides \textsubscript
\ifnum 0\ifxetex 1\fi\ifluatex 1\fi=0 % if pdftex
  \usepackage[T1]{fontenc}
  \usepackage[utf8]{inputenc}
\else % if luatex or xelatex
  \ifxetex
    \usepackage{mathspec}
  \else
    \usepackage{fontspec}
  \fi
  \defaultfontfeatures{Ligatures=TeX,Scale=MatchLowercase}
    \setmonofont[Mapping=tex-ansi,Scale=0.7]{Source Code Pro}
\fi
% use upquote if available, for straight quotes in verbatim environments
\IfFileExists{upquote.sty}{\usepackage{upquote}}{}
% use microtype if available
\IfFileExists{microtype.sty}{%
\usepackage{microtype}
\UseMicrotypeSet[protrusion]{basicmath} % disable protrusion for tt fonts
}{}
\usepackage[margin=1in]{geometry}
\usepackage{hyperref}
\PassOptionsToPackage{usenames,dvipsnames}{color} % color is loaded by hyperref
\hypersetup{unicode=true,
            pdftitle={Statistical Genetics Analyses in R},
            pdfauthor={Dr.~Shirin Glander},
            colorlinks=true,
            linkcolor=Maroon,
            citecolor=Blue,
            urlcolor=Blue,
            breaklinks=true}
\urlstyle{same}  % don't use monospace font for urls
\usepackage{longtable,booktabs}
\usepackage{graphicx,grffile}
\makeatletter
\def\maxwidth{\ifdim\Gin@nat@width>\linewidth\linewidth\else\Gin@nat@width\fi}
\def\maxheight{\ifdim\Gin@nat@height>\textheight\textheight\else\Gin@nat@height\fi}
\makeatother
% Scale images if necessary, so that they will not overflow the page
% margins by default, and it is still possible to overwrite the defaults
% using explicit options in \includegraphics[width, height, ...]{}
\setkeys{Gin}{width=\maxwidth,height=\maxheight,keepaspectratio}
\IfFileExists{parskip.sty}{%
\usepackage{parskip}
}{% else
\setlength{\parindent}{0pt}
\setlength{\parskip}{6pt plus 2pt minus 1pt}
}
\setlength{\emergencystretch}{3em}  % prevent overfull lines
\providecommand{\tightlist}{%
  \setlength{\itemsep}{0pt}\setlength{\parskip}{0pt}}
\setcounter{secnumdepth}{5}
% Redefines (sub)paragraphs to behave more like sections
\ifx\paragraph\undefined\else
\let\oldparagraph\paragraph
\renewcommand{\paragraph}[1]{\oldparagraph{#1}\mbox{}}
\fi
\ifx\subparagraph\undefined\else
\let\oldsubparagraph\subparagraph
\renewcommand{\subparagraph}[1]{\oldsubparagraph{#1}\mbox{}}
\fi

%%% Use protect on footnotes to avoid problems with footnotes in titles
\let\rmarkdownfootnote\footnote%
\def\footnote{\protect\rmarkdownfootnote}

%%% Change title format to be more compact
\usepackage{titling}

% Create subtitle command for use in maketitle
\newcommand{\subtitle}[1]{
  \posttitle{
    \begin{center}\large#1\end{center}
    }
}

\setlength{\droptitle}{-2em}
  \title{Statistical Genetics Analyses in R}
  \pretitle{\vspace{\droptitle}\centering\huge}
  \posttitle{\par}
  \author{Dr.~Shirin Glander}
  \preauthor{\centering\large\emph}
  \postauthor{\par}
  \predate{\centering\large\emph}
  \postdate{\par}
  \date{2017-04-02}


\begin{document}
\maketitle

{
\hypersetup{linkcolor=black}
\setcounter{tocdepth}{1}
\tableofcontents
}
\listoftables
\listoffigures
\textless{}\textgreater{}= options(tikzDefaultEngine=`xetex') @

\chapter*{Preface}\label{preface}
\addcontentsline{toc}{chapter}{Preface}

\section*{Why read this book}\label{why-read-this-book}
\addcontentsline{toc}{section}{Why read this book}

\section*{Structure of the book}\label{structure-of-the-book}
\addcontentsline{toc}{section}{Structure of the book}

\section*{Software information and
conventions}\label{software-information-and-conventions}
\addcontentsline{toc}{section}{Software information and conventions}

All analyses are run on R version 3.3.2 (2016-10-31) -- ``Sincere
Pumpkin Patch'' on a x86\_64-w64-mingw32/x64 (64-bit) platform.

Some packages are available via
\href{https://cran.r-project.org/}{CRAN}, while others are hosted at
\href{https://www.bioconductor.org/}{Bioconductor}. I will provide
package installation instructions at the beginning of each example,
indicating where each package can be found.

I will also be using the \textbf{library()} function, rather than
\textbf{require()} for loading packages to make sure that we will get an
error message in case packages have not been installed correctly.

The example workflows included are meant to illustrate the theoretical
concepts and get you started on your own analysis. They are minimal
examples of the necessary steps but are not meant to substitute the
package manuals. When you want to apply the workflows to your own data,
I highly recommend going back to the package documentation to find out
about additional functions and using the help() function to explore
parameter options. I will be using the same naming and code schemes as
in the package manuals to make finding the relevant parts easy.

\section*{Acknowledgments}\label{acknowledgments}
\addcontentsline{toc}{section}{Acknowledgments}

\chapter*{About the Author}\label{about-the-author}
\addcontentsline{toc}{chapter}{About the Author}

\chapter{Introduction to Statistical
Genetics}\label{introduction-to-statistical-genetics}

\chapter{Quantitative Genetics}\label{quantitative-genetics}

\chapter{Population Genetics}\label{population-genetics}

\chapter{Evolutionary Genetics}\label{evolutionary-genetics}

\chapter{Genetics of complex
diseases}\label{genetics-of-complex-diseases}

\chapter{Genetic Epidemiology}\label{genetic-epidemiology}

\chapter{Placeholder}\label{placeholder}


\end{document}
